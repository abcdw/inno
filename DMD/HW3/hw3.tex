\documentclass[a4paper,12pt]{article}

\usepackage[unicode,colorlinks=true,linkcolor=blue]{hyperref}
\usepackage{amsmath,amssymb}
\usepackage[utf8]{inputenc}
\usepackage[T2A]{fontenc}
\usepackage[russian]{babel}
\usepackage{graphicx}
\usepackage[margin=1in]{geometry}
\usepackage{fancyhdr}

\pagestyle{fancy}
\makeatletter
\fancyhead[L]{\footnotesize BS3-5, 2015, <<DMD>>}
\fancyfoot[L]{\footnotesize \@author}
\fancyfoot[R]{\thepage}
\fancyfoot[C]{}

\renewcommand{\maketitle}{%
\noindent{\bfseries\scshape\large\@title\ \mdseries\upshape(\@date)}\par
\noindent {\large\itshape\@author}
\vskip 2ex}
\makeatother

\newenvironment{problem}[1]{\par\bigskip\noindent\textbf{Problem #1.}
  \enskip\ignorespaces}{}

\title{DMD homework 3} % Fill the number of the homework
\author{Tropin Andrew \\
  e-mail: \href{mailto:andrewtropin@gmail.com}{andrewtropin@gmail.com} \\
  github: \href{http://github.com/abcdw}{abcdw}}
\date{\today} % Fill the date

\begin{document}
  \maketitle
  \thispagestyle{fancy}


  \begin{problem}{1}
    I renamed some attributes for style reason. \\

    \begin{itemize}
      \item One-to-Many:
      \begin{itemize}
        \item Employee(\underline{EID}, address, salary, name)
        \item Room(\underline{ROID}, period)
        \item Patient(\underline{PID}, name, address, ROID, EID)
        \item Record(\underline{REID}, patient\_id, appointment)
        \item Medicine(\underline{CODE}, price)
        \item Bill(\underline{BID}, CODE, dosage, indications, PID)
      \end{itemize}
      \item Many-to-Many:
      \begin{itemize}
        \item Maintains(\underline{EID}, \underline{REID})
      \end{itemize}
      \item Generalizations(ISA):
      \begin{itemize}
        \item Receptionist(\underline{EID}, adress, salary, name)
        \item Doctor(\underline{EID}, adress, salary, name, specialty)
        \item Nurse(\underline{EID}, adress, salary, name, shift, ROID)
        \item Trainee(\underline{EID}, adress, salary, name, specialty)
        \item Permanent(\underline{EID}, adress, salary, name, specialty)
        \item Visiting(\underline{EID}, adress, salary, name, specialty)
      \end{itemize}
    \end{itemize}

  \end{problem}
  \begin{problem}{2}
    \begin{itemize}
      \item $\Pi_{EID, \ldots}\sigma_{ROID = 107}(Nurse) \cup \Pi_{EID, \ldots} (\sigma_{ROID=107}(Patient) \Join Doctor)$
        % \cup \sigma_{\sigma_{ROID=107}(Doctor \Join Patient) (Doctor)$
      \item $Nurse - \Pi_{EID, \ldots}(Nurse \Join Rooms \Join Patient \Join \sigma_{name = Dr.Alex}(Doctor))$
      \item $\sigma_{salary > MIN(\Pi_{salary}(Doctor))}(Employee)$
      \item $\Pi_{ROID}(Patient \Join Rooms) \Join Rooms$
    \end{itemize}
  \end{problem}
  \begin{problem}{3}
    \begin{itemize}
      \item $\sigma_{salary=MAX(\Pi_{e3.salary}\sigma{e1.salary = MAX(\Pi_{salary}(Employees)) \wedge e1.salary > e2.salary \wedge e2.salary > e3.salary
      \wedge e1.eid \not= e2.eid \not= e3.eid}(\rho_{e1}(Employees) \times \rho_{e2}(Employees) \times \rho_{e3}(Employees)))}(Employees)$ Result is $\{(3)\}$. 

      \item $\Pi_{eid} \sigma_{aid = B1100}(Employees \Join Certified)$. Result is $\{(3), (4)\}$
      \item $\Pi_{flight\#}(Flights) - \Pi_{flight\#}( (\Pi_{flight\#}(Flights) \times \Pi_{eid} (\sigma_{salary > 70000}(Employees \Join Certified))) - \Pi_{flight\#, eid}(\sigma_{range \geq distance}(Aircraft \times Flights) \Join Certified)$. Result is $\{(111), (112), (300)\}$
    \end{itemize}
  \end{problem}

  \begin{problem}{4}
    $[0, m * n]$
  \end{problem}
  \begin{problem}{5}
    \begin{itemize}
      \item R(A, B), S(A, B)
      \item R = \{(1, 2)\}
      \item S = \{(1, 3)\}
      \item $\Pi_{A}(R-S) = \{(1)\}, \Pi_{A}(R) - \Pi_{A}(S) = \{\}$
    \end{itemize}
  \end{problem}
  \begin{problem}{6}
    \begin{itemize}
      \item $R \Join S$ and $\sigma_{R.C=S.C}(R \times S)$ near the same, but return relations with different number of attributes.
    First one returns <A,B,C,D> and the second <A,B,R.C,S.C,D>.
      \item Same. Both of them returns uniq tuples with one attribute C, where exist at least one tuple in each relation(R, S) with value of attribute C.
    \end{itemize}
  \end{problem}


\end{document}
