\documentclass[a4paper,12pt]{article}

\usepackage[utf8]{inputenc}
\usepackage{hyperref}
% \usepackage{indentfirst}

\title{DMD homework 1}
\author{Andrew Tropin}
\date{andrewtropin@gmail.com}

\begin{document}
  % \maketitle
  \begin{center}
    \textbf{DMD homework 1} \\
    Andrew Tropin \\
    \href{mailto:andrewtropin@gmail.com}{andrewtropin@gmail.com} \\
  \end{center}

  I have been programming for 9 years. Participated in the semi-finals
  ACM ICPC. Worked in Yandex as python software developer, after as site
  reliability engineer in search perfomance team.

  First of all I don't trust in proprietary software. That why this paper
  cover only open source DBMS. Also it doesn't cover NoSQL, because
  such review is not for one page.

  \begin{table}[h]
    \centering
    \label{tabel:RDBMS}
    \begin{tabular}{c|c|c|c|c}
      DBMS       & SQL           & user management & scalability   & zero-conf \\
      \hline
      sqlite     & weakly typed  & -               & one file      & 3 \\
      mysql      & many features & +               & client-server & 2 \\
      postgresql & most complete & +               & client-server & 1 \\
    \end{tabular}
    \caption{Comparison of RDBMS}
  \end{table}

  This small table \ref{tabel:RDBMS} helps us to separate in three groups,
  according to scale.

  First group contain sqlite.
  sqlite haven't overhead due to IPC, easy to setup and use, zeroconf but
  has limited support of SQL. This RDBMS is widely-spread and have libraries
  for most popular programming languages.
  It's usefull as test DBMS for developing applications or for creating
  small applications with embeded DB. But it can't be scaled and used in
  Highload projects.

  Second group contain MySQl. Most popular open source. It's implements
  many SQL features, easy to setup in small projects, but can be scaled
  for huge projects. It's slow-developing, but can be used in medium-size
  projects.

  Third group contain postgresql. Most stable, feature-rich and
  production-ready. It has huge community and it is intensively developing.
  This DBMS should be used for real highload-projects.

\begin{thebibliography}{99}
  \bibitem{b1} SQLite official site. \href{https://sqlite.org/}{https://sqlite.org/}
  \bibitem{b2} MySQl official site. \href{https://www.mysql.com/}{https://www.mysql.com/}
  \bibitem{b3} postgresql official site. \href{http://postgresql.org/}{http://postgresql.org/}
\end{thebibliography}

\end{document}
