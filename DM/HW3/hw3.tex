\documentclass[a4paper,12pt]{article}

\usepackage[unicode,colorlinks=true,linkcolor=blue]{hyperref}
\usepackage{amsmath,amssymb}
\usepackage[utf8]{inputenc}
\usepackage[T2A]{fontenc}
\usepackage[russian]{babel}
\usepackage{graphicx}
\usepackage[margin=1in]{geometry}
\usepackage{fancyhdr}

\pagestyle{fancy}
\makeatletter
\fancyhead[L]{\footnotesize BS3-5, 2016, <<DM>>}
\fancyfoot[L]{\footnotesize \@author}
\fancyfoot[R]{\thepage}
\fancyfoot[C]{}

\renewcommand{\maketitle}{%
\noindent{\bfseries\scshape\large\@title\ \mdseries\upshape(\@date)}\par
\noindent {\large\itshape\@author}
\vskip 2ex}
\makeatother

\newenvironment{problem}[1]{\par\bigskip\noindent\textbf{Problem #1.} \newline}{}

\title{DM homework 3} % Fill the number of the homework
\author{Tropin Andrew \\
  e-mail: \href{mailto:andrewtropin@gmail.com}{andrewtropin@gmail.com} \\
  github: \href{http://github.com/abcdw}{abcdw}}
\date{\today} % Fill the date

\begin{document}
  \maketitle
  \thispagestyle{fancy}

  \begin{problem}{1}
    Write negations for each of the following statements:\\

    A - John is six feet tall\\
    B - John weights at least 200 pounds\\
    $A \wedge B$ - John is six feet tall and he weighs at least 200 pounds\\
    $\sim (A \wedge B) = \sim A \; \vee \sim B$\\
    John is not six feet tall or he don't weighs at least 200 pounds.\\

    A - The bus was late\\
    B - Tom’s watch was slow\\
    $A \vee B$ - The bus was late or Tom’s watch was slow\\
    $\sim (A \vee B) = \sim A \; \wedge \sim B$\\
    The bus was not late and Tom's watch was not slow

  \end{problem}

  \begin{problem}{2}
    $(\sim p \vee q) \rightarrow \sim q$
    \begin{tabular}{|c|c|c|c|c|c|}
      \hline          
      $p$ & $q$ & $\sim p$ & $\sim p \vee q$ & $\sim q$ & $(\sim p \vee q) \rightarrow \sim q$\\
      \hline  
      1&1&0&1&0&0\\
      \hline  
      1&0&0&0&1&1\\
      \hline
      0&1&1&1&0&0\\
      \hline
      0&0&1&1&1&1\\
      \hline
    \end{tabular} \\
    
    $(\sim p \wedge q) \rightarrow \sim r$ 
    \begin{tabular}{|c|c|c|c|c|c|c|}
       \hline          
       $p$ & $q$ & $r$ & $\sim p$ & $\sim p \wedge q$ & $\sim r$ & $(\sim p \wedge q) \rightarrow \sim r$\\
       \hline  
       1&1&1&0&0&0&1\\
       \hline  
       1&1&0&0&0&1&1\\
       \hline
       1&0&1&0&0&0&1\\
       \hline
       1&0&0&0&0&1&1\\
       \hline
       0&1&1&1&1&0&0\\
       \hline
       0&1&0&1&1&1&1\\
       \hline
       0&0&1&1&0&0&1\\
       \hline
       0&0&0&1&0&1&1\\
       \hline
     \end{tabular}
  \end{problem}

  \begin{problem}{3}
    p: Grizzly bears have been seen in the area.\\
    q: Hiking is safe on the trail.\\
    r: Berries are ripe along the trail.\\
    \begin{itemize}
    \item Berries are ripe along the trail, but grizzly bears have not been seen in the area. $r \wedge \sim p$
    \item Grizzly bears have not been seen in the area and hiking on the trail is safe, but berries are ripe along the trail. $\sim p \wedge r \wedge r$
    \item If berries are ripe along the trail, hiking is safe if and only if
      grizzly bears have not been seen in the area. $r \rightarrow (q \leftrightarrow p)$
    \end{itemize}
  \end{problem}

  \begin{problem}{4}

    \begin{itemize}
    \item 101 1110, 010 0001 \\
    $101 1110 \vee 010 0001 = 111 1111$\\
    $101 1110 \wedge 010 0001 = 000 0000$\\
    $101 1110 \oplus 010 0001 = 111 1111$\\
    \item 1111 0000, 1010 1010 \\
    $1111 0000 \vee 1010 1010 = 1111 1010$\\
    $1111 0000 \wedge 1010 1010 = 1010 0000$\\
    $1111 0000 \oplus 1010 1010 = 0101 1010$\\  
    \end{itemize}
  \end{problem}

  \begin{problem}{5}
    Given any statement form, is it possible to find a logically equivalent form
    that uses only $\sim$ and $\wedge$? \\
    Any logical formula can be represented in CNF with $\sim$, $\wedge$, $\vee$ operators only. Using De Morgan's low you can remove all $\vee$ operators.
  \end{problem}

  \begin{problem}{6}
    Disjunction is commutative.
    \begin{tabular}{|c|c|c|}
      \hline          
      $p$ & $q$ & $p \vee q$\\
      \hline  
      1&1&1\\
      \hline  
      1&0&1\\
      \hline
      0&1&1\\
      \hline
      0&0&0\\
      \hline
    \end{tabular}
    \begin{tabular}{|c|c|c|}
      \hline          
      $p$ & $q$ & $q \vee p$\\
      \hline  
      1&1&1\\
      \hline  
      1&0&1\\
      \hline
      0&1&1\\
      \hline
      0&0&0\\
      \hline
    \end{tabular}\\\\
    
    Disjunction is associative.
    \begin{tabular}{|c|c|c|c|c|}
      \hline          
      $p$ & $q$ & $r$ & $p \vee q$ & $(p \vee q) \vee r$\\
      \hline  
      1&1&1&1&1\\
      \hline  
      1&1&0&1&1\\
      \hline
      1&0&1&1&1\\
      \hline
      1&0&0&1&1\\
      \hline
      0&1&1&1&1\\
      \hline
      0&1&0&1&1\\
      \hline
      0&0&1&0&1\\
      \hline
      0&0&0&0&0\\
      \hline
    \end{tabular}
    \begin{tabular}{|c|c|c|c|c|}
      \hline          
      $p$ & $q$ & $r$ & $q \vee r$ & $p \vee (q \vee r)$\\
      \hline  
      1&1&1&1&1\\
      \hline  
      1&1&0&1&1\\
      \hline
      1&0&1&1&1\\
      \hline
      1&0&0&1&1\\
      \hline
      0&1&1&1&1\\
      \hline
      0&1&0&0&1\\
      \hline
      0&0&1&1&1\\
      \hline
      0&0&0&0&0\\
      \hline
    \end{tabular}\\\\
  \end{problem}

  \begin{problem}{7}
    $\sim (p \wedge q) \not= \sim p \; \wedge \sim q$
    \begin{tabular}{|c|c|c|c|}
      \hline          
      $p$ & $q$ & $p \wedge q$ & $\sim(p \wedge q)$\\
      \hline  
      1&1&1&0\\
      \hline  
      1&0&0&1\\
      \hline
      0&1&0&1\\
      \hline
      0&0&0&1\\
      \hline
    \end{tabular}
    \begin{tabular}{|c|c|c|c|c|}
      \hline          
      $p$ & $q$ & $\sim p$ & $\sim q$ & $\sim p \; \wedge \sim q$\\
      \hline  
      1&1&0&0&0\\
      \hline  
      1&0&0&1&0\\
      \hline
      0&1&1&0&0\\
      \hline
      0&0&1&1&1\\
      \hline
    \end{tabular}\\\\
    
    $\sim (p \wedge q) = \sim p \; \vee \sim q$
    \begin{tabular}{|c|c|c|c|}
      \hline          
      $p$ & $q$ & $p \wedge q$ & $\sim(p \wedge q)$\\
      \hline  
      1&1&1&0\\
      \hline  
      1&0&0&1\\
      \hline
      0&1&0&1\\
      \hline
      0&0&0&1\\
      \hline
    \end{tabular}
    \begin{tabular}{|c|c|c|c|c|}
      \hline          
      $p$ & $q$ & $\sim p$ & $\sim q$ & $\sim p \; \vee \sim q$\\
      \hline  
      1&1&0&0&0\\
      \hline  
      1&0&0&1&1\\
      \hline
      0&1&1&0&1\\
      \hline
      0&0&1&1&1\\
      \hline
    \end{tabular}\\\\

  \end{problem}


  \begin{problem}{8}
    Implication is logically equivalent to its contrapositive. \\
    \begin{tabular}{|c|c|c|}
      \hline          
      $p$ & $q$ & $p \rightarrow q$\\
      \hline  
      1&1&1\\
      \hline  
      1&0&0\\
      \hline
      0&1&1\\
      \hline
      0&0&1\\
      \hline
    \end{tabular}
    \begin{tabular}{|c|c|c|c|c|}
      \hline          
      $p$ & $q$ & $\sim q$ & $\sim p$ & $\sim q \rightarrow \sim p$\\
      \hline  
      1&1&0&0&1\\
      \hline  
      1&0&1&0&0\\
      \hline
      0&1&0&1&1\\
      \hline
      0&0&1&1&1\\
      \hline
    \end{tabular}\\\\
    Converse is logically equivalent to the inverse. \\
    \begin{tabular}{|c|c|c|}
      \hline          
      $p$ & $q$ & $q \rightarrow p$\\
      \hline  
      1&1&1\\
      \hline  
      1&0&1\\
      \hline
      0&1&0\\
      \hline
      0&0&1\\
      \hline
    \end{tabular}
    \begin{tabular}{|c|c|c|c|c|}
      \hline          
      $p$ & $q$ & $\sim p$ & $\sim q$ & $\sim p \rightarrow \sim q$\\
      \hline  
      1&1&0&0&1\\
      \hline  
      1&0&0&1&1\\
      \hline
      0&1&1&0&0\\
      \hline
      0&0&1&1&1\\
      \hline
    \end{tabular}\\\\
  \end{problem}

\end{document}
