\documentclass[a4paper,12pt]{article}

\usepackage[unicode,colorlinks=true,linkcolor=blue]{hyperref}
\usepackage{amsmath,amssymb}
\usepackage[utf8]{inputenc}
\usepackage[T2A]{fontenc}
\usepackage[russian]{babel}
\usepackage{graphicx}
\usepackage[margin=1in]{geometry}
\usepackage{fancyhdr}

\pagestyle{fancy}
\makeatletter
\fancyhead[L]{\footnotesize BS3-5, 2016, <<DM>>}
\fancyfoot[L]{\footnotesize \@author}
\fancyfoot[R]{\thepage}
\fancyfoot[C]{}

\renewcommand{\maketitle}{%
\noindent{\bfseries\scshape\large\@title\ \mdseries\upshape(\@date)}\par
\noindent {\large\itshape\@author}
\vskip 2ex}
\makeatother

\newenvironment{problem}[1]{\par\bigskip\noindent\textbf{Problem #1.} \newline}{}

\title{DM homework 2} % Fill the number of the homework
\author{Tropin Andrew \\
  e-mail: \href{mailto:andrewtropin@gmail.com}{andrewtropin@gmail.com} \\
  github: \href{http://github.com/abcdw}{abcdw}}
\date{\today} % Fill the date

\begin{document}
  \maketitle
  \thispagestyle{fancy}

  \begin{problem}{1}
    Let $M = (8, 15),\ N = (9, 20)$ on real axis, then $K = M \cup N$ will be
    $(8, 20)$
  \end{problem}

  \begin{problem}{2}
    $A = \{a,b,\{a,b\}\}$. How many elements does this set have: $P(A)$.
    \begin{eqnarray*}
      N(A) = 3 \\
      N(P(A)) = 2^{N(A)} = 2^3 = 8
    \end{eqnarray*}
  \end{problem}

  \begin{problem}{3}
    \begin{itemize}
    \item For every $x$ in $\mathbb{R}$ $x$ in power of two not equal to minus one. It's true.
    \item Exists such $x$ in $\mathbb{Z}$, that $x$ in power of two equal to 2.
      It's false.
    \end{itemize}
  \end{problem}

  \begin{problem}{4}
    Let $A = \{1,2,3,4,5\}$ and $B = \{0,3,6\}$. Find
    \begin{itemize}
    \item $A \cup B = \{0,1,2,3,4,5,6\}$
    \item $A \cap B = \{3\}$
    \item $A - B = \{1, 2, 4, 5\}$
    \item $B - A = \{0, 6\}$
    \end{itemize}
  \end{problem}

  \begin{problem}{5}
    $A - B = A \cap B^c$
    \begin{eqnarray*}
      x \in (A - B) \Rightarrow
      x \in A \& x \not\in B \Rightarrow
      x \in A \& x \in B^c \Rightarrow
      x \in (A \cap B^c)
    \end{eqnarray*}
    \begin{eqnarray*}
      x \in (A \cap B^c) \Rightarrow
      x \in A \& x \in B^c \Rightarrow
      x \in A \& x \not\in B \Rightarrow
      x \in (A - B)
    \end{eqnarray*}
  \end{problem}

  \begin{problem}{6}
    \begin{itemize}
    \item $A \cap B \cap C = \{4, 6\}$
    \item $(A \cup B) \cap C = \{4, 5, 6, 8, 10\}$
    \end{itemize}
  \end{problem}

  \begin{problem}{7}
    It's hard enough to draw them with LaTeX.
    \begin{itemize}
    \item $A \cap (B - C)$ will look like three circles and only one small
      part, that belongs to A and B, but not C will be filled.
    \item All universe will be filled, except circles related to $A, B, C$.
    \end{itemize}
  \end{problem}


  \begin{problem}{8}
    $A^c \cup (A \cup B^c \cup C^c)^c \cup (B \cap (A \cup C)^c)
    = A^c \cup (B - (A \cup C)) = A^c$
  \end{problem}

\end{document}
