\documentclass[a4paper,12pt]{article}

\usepackage[unicode,colorlinks=true,linkcolor=blue]{hyperref}
\usepackage{amsmath,amssymb}
\usepackage[utf8]{inputenc}
\usepackage[T2A]{fontenc}
\usepackage[russian]{babel}
\usepackage{graphicx}
\usepackage[margin=1in]{geometry}
\usepackage{fancyhdr}

\pagestyle{fancy}
\makeatletter
\fancyhead[L]{\footnotesize BS3-5, 2016, <<DM>>}
\fancyfoot[L]{\footnotesize \@author}
\fancyfoot[R]{\thepage}
\fancyfoot[C]{}

\renewcommand{\maketitle}{%
\noindent{\bfseries\scshape\large\@title\ \mdseries\upshape}\par
\noindent {\large\itshape\@author}
\vskip 2ex}
\makeatother

\renewcommand{\theenumi}{\alph{enumi}}
\newenvironment{problem}[1]{\par\bigskip\noindent\textbf{Problem #1.} \newline}{}

\title{DM homework 5} % Fill the number of the homework
\author{Tropin Andrew \\
  e-mail: \href{mailto:andrewtropin@gmail.com}{andrewtropin@gmail.com} \\
  github: \href{http://github.com/abcdw}{abcdw}}
\date{\today} % Fill the date

\begin{document}
  \maketitle
  \thispagestyle{fancy}

  \begin{problem}{1}
    $\frac{x^8+x^4-2x^2+6}{x^4+2x^2+3} + 2x^2 - 2 = x^4$\\
    $x^4$ can't be prime, sorry :'(
  \end{problem}

  \begin{problem}{2}
    Mistake is that we took as free variable k in both numbers (n and m).
    Difference between odd and even number is not always 1.
  \end{problem}

  \begin{problem}{3}
    $\frac12 + 8(n^2(2n^2 + 3) + 1) - \frac{\cos 2n}{2}+\frac{1}{1+\tan^2n} =\\
    16n^4 + 24n^2 + 8 + \sin^2n + \cos^2n = 16n^4 + 24n^2 + 9 = (4n^2 + 3)^2$
  \end{problem}

  \begin{problem}{4}
    $r = \fracpq$ is rational if p and $q\not=0$ are integers.
    Multiplication and addition of integers produce integers.
    That's why $10m+ 15n$ and $4n$ are integers. That's mean, that
    $\frac{10m+15n}{4n}$ is rational by definition.
  \end{problem}

  \begin{problem}{5}
    $x^2 + bx + c = 0$\\
    $d = b^2 - 4c$\\
    $x_{12} = \frac{-b \pm \sqrt{d}}{2}$\\
    If $x_1$ is rational then $-b + \sqrt{d}$ is integer and if $-b +\sqrt{d}$
    is integer then $-b - \sqrt{d}$ is also integer or vise verse. And that's
    mean that $x_2$ is also rational.
  \end{problem}

  \begin{problem}{6}
    Let $r_i = \frac{p_i}{q_i}$. Just multiply first equation by $q_3q_2q_1q_0$
    and get integer numbers instead of rational.
  \end{problem}

  \begin{problem}{7}
    Lets do some magic and we will get that
    $\frac1a = \frac1x + \frac1y$\\
    $\frac1b = \frac1x + \frac1z$\\
    $\frac1c = \frac1y + \frac1z$\\
    $x = 1/(\frac1{2a} + \frac1{2b} - \frac1{2c})$ x is rational.
  \end{problem}


  \begin{problem}{9}
    Just move condition from while(condition) loop to for(;condition;) loop. And
    do same thing with body of the loop.
  \end{problem}
  
\end{document}
