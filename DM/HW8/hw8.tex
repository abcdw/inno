\documentclass[a4paper,12pt]{article}

\usepackage[unicode,colorlinks=true,linkcolor=blue]{hyperref}
\usepackage{amsmath,amssymb}
\usepackage[utf8]{inputenc}
\usepackage[T2A]{fontenc}
\usepackage[russian]{babel}
\usepackage{graphicx}
\usepackage[margin=1in]{geometry}
\usepackage{fancyhdr}

\pagestyle{fancy}
\makeatletter
\fancyhead[L]{\footnotesize BS3-5, 2016, <<DM>>}
\fancyfoot[L]{\footnotesize \@author}
\fancyfoot[R]{\thepage}
\fancyfoot[C]{}

\usepackage{minted}

\renewcommand{\maketitle}{%
\noindent{\bfseries\scshape\large\@title\ \mdseries\upshape(\@date)}\par
\noindent {\large\itshape\@author}
\vskip 2ex}
\makeatother

\renewcommand{\theenumi}{\alph{enumi}}
\newenvironment{problem}[1]{\par\bigskip\noindent\textbf{Problem #1.} \newline}{}

\title{DM homework 8} % Fill the number of the homework
\author{Tropin Andrew \\
  e-mail: \href{mailto:andrewtropin@gmail.com}{andrewtropin@gmail.com} \\
  github: \href{http://github.com/abcdw}{abcdw}}
\date{\today} % Fill the date

\begin{document}
  \maketitle
  \thispagestyle{fancy}

  \begin{problem}{1}
    $g(A) = \{a\}, g(X) = \{a, d\}.$\\
    Inverse function doesn't exists. But maybe
    you want something like this: $ g^{-1}(C) = \{1, 2, 3\}, g^{-1}(D) =
    \emptyset, g^{-1}(Y) = X$
  \end{problem}

  \begin{problem}{2}
    \begin{enumerate}
    \item true
    \item false, because inverse function may not exists.
    \end{enumerate}
  \end{problem}
  
  \begin{problem}{4}
    \begin{enumerate}
    \item They one to one, same result can't be reached with different
      parametrs, because of primes.
    \item They are not onto. 2 can't be reached.
    \end{enumerate}
  \end{problem}

  \begin{problem}{5}
    Because of fulfillment of two properties these functions both bijections.
    Bijections always have inverse function. If functions is bijections then
    composition is bijection too. As we said above bijections always have
    inverse function. 
    But actually $f(g(x))$ doesn't exists because of incompatible domains.
  \end{problem}

  \begin{problem}{8}
    \inputminted{python}{calc.py}
  \end{problem}
\end{document}

%%% Local Variables:
%%% coding: utf-8
%%% TeX-command-extra-options: "-shell-escape"
%%% End: