\documentclass[a4paper,12pt]{article}

\usepackage[unicode,colorlinks=true,linkcolor=blue]{hyperref}
\usepackage{amsmath,amssymb}
\usepackage[utf8]{inputenc}
\usepackage[T2A]{fontenc}
\usepackage[russian]{babel}
\usepackage{graphicx}
\usepackage[margin=1in]{geometry}
\usepackage{fancyhdr}

\pagestyle{fancy}
\makeatletter
\fancyhead[L]{\footnotesize BS3-5, 2016, <<DM>>}
\fancyfoot[L]{\footnotesize \@author}
\fancyfoot[R]{\thepage}
\fancyfoot[C]{}

\renewcommand{\maketitle}{%
\noindent{\bfseries\scshape\large\@title\ \mdseries\upshape(\@date)}\par
\noindent {\large\itshape\@author}
\vskip 2ex}
\makeatother

\newenvironment{problem}[1]{\par\bigskip\noindent\textbf{Problem #1.} \newline}{}

\title{DM homework 1} % Fill the number of the homework
\author{Tropin Andrew \\
  e-mail: \href{mailto:andrewtropin@gmail.com}{andrewtropin@gmail.com} \\
  github: \href{http://github.com/abcdw}{abcdw}}
\date{\today} % Fill the date

\begin{document}
  \maketitle
  \thispagestyle{fancy}

  \begin{problem}{1}
    Prove or disprove:

    \begin{enumerate}
    \item if $a*b = a$: prove that $b = 1$ \\
      \begin{eqnarray*}
        \label{eq:problem1i1}
        a^{-1} * a * b &=& a^{-1} * a \\
        1 * b &=& 1;\ 1 * b = b \\
        b &=& 1
      \end{eqnarray*}
    \item The difference of any two odd integers is even. \\
      \begin{eqnarray*}
        \label{eq:problem1i2}
        n_1 &=& 2 * k + 1\\
        n_2 &=& 2 * m + 1 \\
        n_1 - n_2 &=& 2 * k + 1 - (2 * m + 1) \\
        n_1 - n_2 &=& 2 * (k - m)
      \end{eqnarray*}
    \end{enumerate}
  \end{problem}

  \begin{problem}{2}
    Prime numbers:

    \begin{enumerate}
    \item Is $n^k - 1$ prime for any integers n and k? \\
      Nope, $1$ isn't integer($n = 2,\ k = 1$).
    \item Is expression $n^2 - n + 41$ a prime number? \\
      Nope, if we take $n = k * 41\ (k \not= 0)$, that expression won't be equal
      to 41 and will be divisible at least by 41.
    \end{enumerate}
  \end{problem}
  
  \begin{problem}{3}
    Divisibility:

    \begin{enumerate}
    \item Prove that sum of $2*n + 1$ consecutive numbers is divisible by $2 * n
      + 1$\
      \begin{eqnarray*}
        \label{eq:problem3i1}
        a = k,\ b = k + 2n + 1 - 1\\
        \sum_{x=a}^b x = \frac{(a + b)}{2} * (b - a + 1) = \frac{(k + k + 2n)(2n + 1)}{2} = (k + n)(2n + 1)
      \end{eqnarray*}
    \item Find quotient and divisor of:
      \begin{itemize}
      \item $n^3+2n-1$ divided by $n$
        $$
        \arraycolsep=0.05em
        \begin{array}{rrr@{\,}r|r}
          n^3&{}+2n&{}-12&&\,n\\
          \cline{5-5}
          n^3&{}&&&\,n^2 + 2\\
          \cline{1-2}
          &{}2n&{}-12\\
          &{}2n&{}\\
          \cline{2-3}
          &{}&{}-12
        \end{array}
        $$
        divisor: $n^2 + 2$, quotient: $-12$
      \item $12n^5+10n^4+2$ divided by $2n+1$
        $$
        \arraycolsep=0.05em
        \begin{array}{rrr@{\,}r|r}
          12n^5&{} + 10n^4&{}+2&&\,2n+1\\
          \cline{5-5}
          12n^5&{} + 6n^4&&&\,6n^4 + 2n^3 - n^2 + \frac{1}{2}n - \frac14\\
          \cline{1-2}
          &{}4n^4&{}+2\\
          &{}4n^4&{} + 2n^3\\
          \cline{2-3}
          &{}-2n^3&{} + 2\\
          &{}-2n^3&{} - n^2\\
          \cline{2-3}
          &{}n^2&{}+2\\
          &{}n^2&{}+\frac{n}{2}\\
          \cline{2-3}
          &{}-\frac{n}{2}&{} + 2\\
          &{}-\frac{n}{2}&{} - \frac14\\
          \cline{2-3}
          &{}&{}\frac94
        \end{array}
        $$
        divisor: $6n^4 + 2n^3 - n^2 + \frac{1}{2}n - \frac14$, quotient: $\frac94$
      \end{itemize}
    \end{enumerate}
  \end{problem}

  \begin{problem}{4}
    Write each rational number as a ratio of two integers:
    \begin{itemize}
    \item $a = 0.4\overline{6271}$ \\
      \begin{eqnarray*}
        a * 10000 - a &=& 4627.1... - 0.4... \\
        9999a &=& 4626.7\\
        a &=& \frac{46267}{99990}
      \end{eqnarray*}
    \item $b = 12.1\overline{121}$
      \begin{eqnarray*}
        b * 1000 - b &=& 12112.1... - 12.1... \\
        999b &=& 12100\\
        b &=& \frac{12100}{999}
      \end{eqnarray*}
    \end{itemize}
  \end{problem}
  
  \begin{problem}{5}
    \begin{enumerate}
    \item If $r$ is any rational number, then $3r^2 - 2r + 4$ is rational. \\
      \begin{eqnarray*}
        r &=& \frac{p}{q},\ p \in \mathbb{Z},\ q \in \mathbb{N} \backslash \{0\} \\
        \frac{3p^2}{q^2} - 2\frac{p}{q} + 4 &=& \frac{3p^2 - 2pq + 4q^2}{q^2} =
        \frac{a}{b},\ a \in \mathbb{Z},\ b \in \mathbb{N}
      \end{eqnarray*}
    \item Product and sum of two rational numbers is rational. \\
      \begin{eqnarray*}
        a = \frac{n}{m},\ b = \frac{p}{q},\ n,p \in \mathbb{Z},\ m,q \in \mathbb{N} \\
        a * b = \frac{nm}{pq},\ nm \in \mathbb{Z},\ pq \in \mathbb{N} \\
        a + b = \frac{nq + mp}{pq},\ (nq + mp) \in \mathbb{Z},\ pq \in \mathbb{N} \\

      \end{eqnarray*}
    \end{enumerate}
  \end{problem}
\end{document}
