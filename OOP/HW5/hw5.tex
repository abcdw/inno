\documentclass[a4paper,12pt]{article}

\usepackage[unicode,colorlinks=true,linkcolor=blue]{hyperref}
\usepackage{amsmath,amssymb}
\usepackage[utf8]{inputenc}
\usepackage[T2A]{fontenc}
\usepackage[russian]{babel}
\usepackage{graphicx}
\usepackage[margin=1in]{geometry}
\usepackage{fancyhdr}
\usepackage{minted}

\pagestyle{fancy}
\makeatletter
\fancyhead[L]{\footnotesize BS3-5, 2015, <<OOP>>}
\fancyfoot[L]{\footnotesize \@author}
\fancyfoot[R]{\thepage}
\fancyfoot[C]{}

\renewcommand{\maketitle}{%
\noindent{\bfseries\scshape\large\@title\ \mdseries\upshape(\@date)}\par
\noindent {\large\itshape\@author}
\vskip 2ex}
\makeatother

\newenvironment{problem}[1]{\par\bigskip\noindent\textbf{Problem #1.}
  \enskip\ignorespaces}{}

\title{OOP homework 5} % Fill the number of the homework
\author{Tropin Andrew \\
  e-mail: \href{mailto:andrewtropin@gmail.com}{andrewtropin@gmail.com} \\
  github: \href{http://github.com/abcdw}{abcdw}}
\date{\today} % Fill the date

\begin{document}
  \maketitle
  \thispagestyle{fancy}


  \begin{problem}{1}
    \begin{minted}{eiffel}
note
	description: "Creating new objects for Zurich."

class
	OBJECT_CREATION

inherit
	ZURICH_OBJECTS

feature -- Explore Zurich

	explore
			-- Create new objects for Zurich.
		do
			add_buildings
			add_route
		end
	add_buildings
		local
			first_corner, second_corner: VECTOR
			eth, opera: BUILDING
		do
		    create first_corner.make(100, -100)
		    create second_corner.make(200, -200)
		    create eth.make ("Super Street", first_corner, second_corner)

		    create first_corner.make (400, -1000)
		    create second_corner.make (300, -1100)
		    create opera.make ("Super Street 2", first_corner, second_corner)

		    opera.set_name("Opera")
		    eth.set_name("ETH")
		    Zurich.add_building(eth)
		    Zurich.add_building(opera)
		end

		add_route
			local
				leg1, leg2, leg3: LEG
				opera_route: ROUTE
			do
			    create leg1.make (Zurich.station("Polyterrasse"), Zurich.station("Central"), Zurich.line(24))
			    create leg2.make (Zurich.station("Central"), Zurich.station("Paradeplatz"), Zurich.line(7))
			    create leg3.make (Zurich.station("Paradeplatz"), Zurich.station("Opernhaus"), Zurich.line(2))
			    leg1.link(leg2)
			    leg2.link(leg3)
			    create opera_route.make(leg1)
			    Zurich.add_route(opera_route)
			end
end
    \end{minted}
  \end{problem}

  \begin{problem}{2}
    \begin{minted}{eiffel}
note
	description: "Temperature."

class
	TEMPERATURE

create
	make_celsius, make_kelvin

feature -- Initialization

	make_celsius (v: INTEGER)
			-- Create with Celsius value `v'.
		require
			above_absolute_zero: v >= -Celsius_zero
		do
			celsius := v
		ensure
		    celsius_value_set: celsius = v
			-- Create a temperature object encapsulating value 'v' intended in Celsius.
			-- Your code here
		end

	make_kelvin (v: INTEGER)
			-- Create with Kelvin value `v'.
		require
			above_absolute_zero: v >= 0
		do
			celsius := v - Celsius_zero
		ensure
		    kelvin_value_set: kelvin = v
			-- Your code here
			-- Create a temperature object encapsulating value 'v' intended in Kelvin.
		end

feature -- Access

	Celsius_zero: INTEGER = 273

	celsius: INTEGER
			-- Value in Celsius scale.

	kelvin: INTEGER
			-- Value in Kelvin scale.
		do
			Result := celsius + Celsius_zero
			-- Your code here
			-- Compute the Kelvin temperature value from the Celsius value
		end

feature -- Measurement

	average (other: TEMPERATURE): TEMPERATURE
			-- Average temperature between `Current' and `other'.
		require
		    other_exists: other /= Void
		do
			create Result.make_celsius((celsius + other.celsiuss) // 2)
			ensure
				between: (celsius <= Result.celsius and Result.celsius <= other.celsius) or
				(other.celsius <= Result.celsius and Result.celsius <= celsius)
			-- Your code here.
			-- Compute the average of two temperature. One is given by the current object,
			-- the other is passed as an argument.
		end
invariant
    above_absolute_zero: kelvin >= 0
end
    \end{minted}
    \begin{minted}{eiffel}
note
	description : "project application root class"
class
	APPLICATION

inherit
	ARGUMENTS

create
	make

feature {NONE} -- Initialization

	make
		local
		    t1, t2, t3: TEMPERATURE
			-- Run application.
		do
			--| Add your code here
			-- print ("Hello Eiffel World!%N")
			print ("Enter t1 in Celsius: ")
			Io.read_integer
			create t1.make_celsius (Io.last_integer)
			print ("t1 in Kelvin is: ")
			print (t1.kelvin)
			print ("%N")

			print ("Enter t2 in Kelvin: ")
			Io.read_integer
			create t2.make_kelvin(Io.last_integer)
			print ("t2 in Celsius is: ")
			print (t2.celsius)
			print ("%N")

			t3 := t1.average(t2)
			print ("Average in Celsius is: ")
			print (t3.celsius)
			print ("%N")
			print ("Average in Kelvin is: ")
			print (t3.kelvin)
			print ("%N")
		end

end
    \end{minted}
  \end{problem}

\end{document}
